\section{Project Description}
\kommentar{Bare direkte indsat fra projektbeskrivelsen vi fik udleveret}
The overall goal of this project is to develop a scalable, high performance search engine. The main focus is on the algorithmic challenges in compactly representing a large data-set while supporting fast searches on it. 

The project consists of a basic part and an advanced part. The basic part corresponds to at approximately 3 ECTS. 

Teaching Goals
\begin{itemize}
    \item Implement relevant data structures and algorithms. 
    \item Analyze and evaluate the efficiency of the solutions from a theoretical and practical perspective, using both theoretical asymptotic analysis and empirical analysis.
    \item Document key relevant aspects of the work in a concise manner. 
\end{itemize}

\section{Problem analysis}

Building an effective search engine is a challenging task that requires efficient algorithms and data structures to manage large amounts of data. Moreover, the algorithms and data structures vary greatly depending on which functionalities the search engine has to implement. The first step is to support queries such as "Which documents contain the word X?". For this part, it is enough to store each word in a simple data structure and look up which document they appear in. Two straight forwards approaches are to use linked lists or hash functions. More advanced queries prompt the search engine for similarly more advanced data structures. Queries such as "Which documents contain the words shoe and boot but not sandal?", "Which documents contain a word starting with be?", and "Which documents contain the sentence 'to be or not to be'?" each require their own additions. The challenge is to support as many of these queries as possible while compactly representing the dataset and supporting fast searches. 

%However, when working with a large dataset, managing and searching for data can be challenging. The implementations of search engines using linked lists or hash functions are therefore just a starting point.


