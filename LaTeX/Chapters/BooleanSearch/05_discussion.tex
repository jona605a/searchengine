\section{Discussion}
When inspecting figure \ref{fig:Searchtimebool1} and figure \ref{fig:Searchtimebool7} it is important to keep in mind that the timed search function both consists of an evaluation function, that takes the query and evaluates it, and a decoder, that traverses the encoded article list and returns a list of the titles of the articles, that matches the query.

To summarise and compare the indexes, the number of computational steps of each boolean operation, when evaluating a boolean query of depth 1 and decoding the resulting article list, have been gathered in table\ref{tab:Booleanruntimes1}. The run-time complexities of the operations are gathered in table \ref{tab:BooleanruntimesO}. The purpose of table \ref{tab:Booleanruntimes1} is, to estimate the number of steps required to perform the operation and thereby compare the expected coefficient of the linear growths.

%An article list $A_n$ is upper bounded by $O(a)$. This means that all operations in table \ref{tab:Booleanruntimes1} has the time complexity $O(a)$ except for the AND operation for index 8.2 and 8.3, which is bounded by $O(min(|A_i| + |A_j|, log(|A_i|)\cdot |A_j|))$, and $\lceil \frac{a}{64} \rceil$ and $a$ which has the tight bound $\Theta(a)$. 

\begin{table}[H]
\begin{tabular}{l|llll}
    Index & AND & OR & INVERSION & Decoding of article list\\
\hline
7.0 &  $\lceil \frac{a}{64} \rceil$   &  $\lceil \frac{a}{64} \rceil$  &  $\lceil \frac{a}{64} \rceil$ & $a$         \\
8.0 &  $|A_i| + |A_j|$   &  $|A_i| +|A_j|$  &  $|A_i|$  & $|A_{result}|$       \\
8.1 &  $|A_i| + |A_j|$   &  $|A_i| + |A_j|$  &  $|A_i|$  & $|A_{result}|$         \\
8.2 &  $min(|A_i| + |A_j|, log(|A_i|)\cdot |A_j|)$   &  $|A_i| + |A_j|$  &  $|A_i|$  & $|A_{result}|$    \\
8.3 &  $min(|A_i| + |A_j|, log(|A_i|)\cdot |A_j|)$   &  $|A_i| + |A_j|$  &  $|A_i|$  & $|A_{result}|$     \\
8.4 &  $|A_i| + |A_j| + \lceil \frac{a}{64} \rceil$   &   $|A_i| + |A_j| + \lceil \frac{a}{64} \rceil$   &  $|A_i| + \lceil \frac{a}{64} \rceil$   & $a$
\end{tabular}
\caption{Caption}
\label{tab:Booleanruntimes1}
\end{table}

\begin{table}[H]
\begin{tabular}{l|llll}
    Index & AND & OR & INVERSION & Decoding of article list\\
\hline
7.0 &  $\Theta(a)$   &  $\Theta(a)$  &  $\Theta(a)$ & $\Theta(a)$         \\
8.0 &  $O(a)$   &  $O(a)$  &  $O(a)$  & $O(a)$       \\
8.1 &  $O(a)$   &  $O(a)$  &  $O(a)$  & $O(a)$         \\
8.2 &  $O(min(|A_i| + |A_j|, log(|A_i|)\cdot |A_j|))$   &  $O(a)$  &  $O(a)$  & $O(a)$    \\
8.3 &  $O(min(|A_i| + |A_j|, log(|A_i|)\cdot |A_j|))$   &  $O(a)$  &  $O(a)$  & $O(a)$     \\
8.4 &  $\Theta(a)$   &   $\Theta(a)$   &  $\Theta(a)$   & $\Theta(a)$
\end{tabular}
\caption{Caption}
\label{tab:BooleanruntimesO}
\end{table}

In figure \ref{fig:Searchtimebool1} it is seen that Index 7.0  in general faster than the other indexes. All boolean operations of 7.0 does however have a time complexity of $\Theta(a)$. Index 7.0 might therefore perform more steps than the other indexes but because it utilises the bit wise Boolean operations, which are performed very fast, it still ends up being faster thean the other indexes. This does however come at the cost of index 7 being the most expensive index when comparing memory complexity, as it stores the article list using $\theta(a)$ bits. 

Index 8.4 does also use fast bit-wise Boolean operations, but before doing so it needs to convert the article list, to a bit vector article list. When comparing Index 8.0 and 8.4 on queries of depth 1 in table \ref{tab:Booleanruntimes1} it is clear to see that 8.0 is better. Creating the bit-wise article lists costs the same as performing the AND operation of index 8.0 itself. Furthermore the decoding of the article list of index 8.0 upper bounded by $O(a)$ whereas decoding of the article list of index 8.4 is tightly bounded by $\Theta(a)$, as the decoding of the article list of index 8.4 always evaluates $a$ bits.  Index 8.4 does therefore not perform greatly, compared to the other indexes, when the depth only is 1.

However, whenever the depth of the search query increases, index 7.0 and 8.4 is expected to be faster than any of the other indexes. This tendency is also represented in figure \ref{fig:Searchtimebool7}. Why index 7.0 and 8.4 performs greatly at evaluatin query trees of higher depths can be seen by evaluating the steeps they perform to do so.

Table \ref{tab:Booleanruntimesd} shows an estimate for the number of computational steps needed to evaluate a tree of depth $d$, consisting of only either the AND, the OR or the INVERSION operator.

\begin{table}[H]
\begin{tabular}{l|lll}
Index 
& AND                           
& OR                             
& INVERSION \\         
\hline
7.0 
&  $\lceil \frac{a}{64} \rceil \cdot (2^d-1)$   
&  $\lceil \frac{a}{64} \rceil \cdot (2^d-1)$  
&  $\lceil \frac{a}{64} \rceil \cdot d$         \\
8.0 
&  $\sum_{n=1}^{2^{d+1}-2}(|A_n|)$        
&  $\sum_{n=1}^{2^{d+1}-2}(|A_n|)$       
&  $\sum_{n=1}^{d}(|A_n|)$             \\
8.4 
&  $\sum_{n=1}^{2^{d}}(|A_n|) + (2^{d}-1) \cdot \lceil \frac{a}{64} \rceil$   
&  $\sum_{n=1}^{2^{d}}(|A_n|) + (2^{d}-1) \cdot \lceil \frac{a}{64} \rceil$   
&  $A_1 + d \cdot \lceil \frac{a}{64} \rceil$   
\end{tabular}
\caption{Caption}
\label{tab:Booleanruntimesd}
\end{table}

A query tree of depth $d$ consisting of either AND operators or OR operators is a perfectly balanced binary tree. The depth $d$ of the tree is defined as the maximum number of edges from the root to a leaf. There are thereby $2^{d+1}-1$ nodes in total, where $2^d$ of the nodes are leaves and $2^d-1$ nodes are parents. To evaluate the query tree using index 8.0, $2^d-1$ Boolean operations must be performed - one for each parent. Performing one Boolean operation in 8.0 takes the length of the article lists of the children of the node. Thereby $\sum_{n=1}^{2^{d+1}-2}(|A_n|)$ in total - every node expect the root. Whereas 8.4 needs to convert all article lists of children into bit-wise article lists taking $\sum_{n=1}^{2^{d}}(|A_n|)$ steps and hereafter evaluate $2^d-1$ Boolean operations each taking $\lceil \frac{a}{64} \rceil$ time. The difference between 8.0 and 8.4, therefore, comes down to 

\begin{equation} \label{eq1}
\begin{split}
 & \sum_{n=1}^{2^{d+1}-2}(|A_n|) - (\sum_{n=1}^{2^{d}}(|A_n|) + (2^{d}-1) \cdot \lceil \frac{a}{64} \rceil) \\
 & = \sum_{n=2^d}^{2^{d+1}-2}(|A_n|) - ((2^{d}-1) \cdot \lceil \frac{a}{64} \rceil)
\end{split}
\end{equation}

It thereby comes down to if 8.0 evaluates the second half of the query tree (all the nodes in the tree with a height of 2 or more) faster than 8.4.  $\sum_{n=2^d}^{2^{d+1}-2}(|A_n|)$ is upper bounded by $O(a)$ whereas $((2^{d}-1) \cdot \lceil \frac{a}{64} \rceil$ is tightly bounded by $\Theta(a)$. 8.4 is therefore expected to perform more steps than 8.0. 8.4 is however still expected to be faster as it utilises bitwise Boolean operation, which can be performed super fast, compared to 8.0, which compares two characters, adds one of them to a result list, and hereafter moves a pointer. This is the same tendency as seen in figure \ref{fig:Searchtimebool1}, where 7.0 where the fasted index. The performance of 8.0 compared to 8.4 is however greatly dependent on the expected length of the article list in the second half of the query tree. A short article list compared to $\lceil \frac{a}{64} \rceil$ will give 8.0 a great advantage that at some point might outperform the bit-wise operations of 8.4. The length of the article list in the second half of the query tree is however greatly dependent on the boolean operators in the tree. 

The AND operation returns the intersection, $A_{result}$, of the article list of the two children, $A_i$ and $A_j$. This means that $|A_{result}|\leq min(|A_i|,|A_j|)$. The AND operation is therefore expected to run faster at nodes higher in a tree, only constructed by AND operations, as the incoming article list is expected to be shortened. 

The exact opposite is expected for a tree only consisting of OR notes as OR notes calculate the union of two article lists making $|A_result|\geq min(|A_i|,|A_j|)$. This is a great disadvantage for 8.0 and makes 8.4 much better at evaluating boolean trees, with a large $d$, only consisting of OR operators. This exact type of tree will appear in the section Prefix search.

For a Boolean query tree of mixed tree 

\kommentar{Skriv færdigt}

\begin{table}[H]
\begin{tabular}{l|lllllllll}
 Filesize (MB)                           & 0.1 & 1     & 2      & 5      & 10 & 20     & 50     & 100 & 200 \\
 \hline 
Mean length of article list &  1.22   & 2.46     & 3.13      & 4.23      & 5.48      &  6.85     &  9.54    &  12.06      &  15.36      \\
  $\lceil \frac{a}{64} \rceil$                          &  1      & 1        & 2         &  6        & 14      &  28       &  66     & 129     &  282    
\end{tabular}
\caption{Caption}
\label{fig: Searchtimebool7}
\end{table}

Even though index 7.0 and 8.4 performs better than the other indexes as seen in figure \ref{fig: Searchtimebool7}, index 7.0 and 8.4 would rarely be chosen over the other indexes in a real-life setting with large databases. This is due to the fact that they are too expensive in memory. Index 7.0 uses $\Theta(a\cdot u)$ space to simply store the index. Even though 8.4 only uses $\Theta(u)$ space to store the index, the extra space used when searching, will at some point become too expensive, as a $O(a)$ bitwise article list needs to be created pr word in the search query.

Index 8.0, 8.1, 8.2 and 8.3 all perform similarly. This tendency does not change when the depth of the query is increased. This may indicate the small changes in using De-Morgans law or binary search or both not is helpful. The randomly generated queries may not generate enough cases where the additional evaluation features are helpful. The time won may therefore not out weight the extra time it takes to check if the evaluation feature can be used, or are simply not visible in the test that also times many other queries. 

The set up of the timing may not justify the potential of Index 8.1 and 8.2. To show the full potential of these indexes we would in hindsight have liked to create more benchmarking timing specifically designed for these indexes. E.g. timing 1000 examples of evaluating a query on the form $!A \vee ! B$ and $!A \wedge ! B$  compared to $!(A \wedge B)$ and $!(A \vee B)$. Using De-morgans law is theoretically beneficial as it omits one inversion, which is the most expensive Boolean operation. Performing a test, with a focus on this, would therefore be interesting to see how much time is won in practice, and if it would be possible to map the change to performing precisely one inversion less.

It is however worth pointing out that De-morgans law, rarely occurs in in naturally generated search queries, which means that there rarely is any time to win using this method. Checking to see when the method can be used therefore seems inconvenient, as omitting one inversion is negligible when looking at a lot of queries as in figure \ref{fig:Searchtimebool1}.

To justify 8.2, queries, where the binary search approach is helpful, could be constructed. These queries could then be evaluated by either index 8.2 or index 8.0 and compared. In theory 8.2 should perform better once again. Furthermore would it be possible to determine the constant of $c_1,c_2,c_3$ and $c_4$ in $c_1\texttt{len}(A_0) \cdot c_2 log_2(\texttt{len}(A_1)) < c_3 \texttt{len}(A_0) + c_4 \texttt{len}(A_1)$ to make index 8.2 even better at deciding when to use the binary search approach and when not to.

As non of the indexes filters out any stop words, other than grammatical signs, the binary search feature could potentially be very useful as stop words occur in almost all articles, whereas many other words only occur in a few articles. Even though 8.4 performs much better than index 8.2, at the filesizes tested out in these plots, generating the bit-wise article list will at some point become too expensive when $a$ grows. Therefore 8.2 would be interesting to develop further.